\graphicspath{ {images/} }

\titledquestion{Analyzing NMT Systems}[25]

\begin{parts}

    \part[3] Look at the {\monofam{src.vocab}} file for some examples of phrases and words in the source language vocabulary. When encoding an input Mandarin Chinese sequence into ``pieces'' in the vocabulary, the tokenizer maps the sequence to a series of vocabulary items, each consisting of one or more characters (thanks to the {\monofam{sentencepiece}} tokenizer, we can perform this segmentation even when the original text has no white space). Given this information, how could adding a 1D Convolutional layer after the embedding layer and before passing the embeddings into the bidirectional encoder help our NMT system? \textbf{Hint:} each Mandarin Chinese character is either an entire word or a morpheme in a word. Look up the meanings of 电, 脑, and 电脑 separately for an example. The characters 电 (electricity) and  脑 (brain) when combined into the phrase 电脑 mean computer.

    \ifans{The hint presents a very important point: the meaning of characters can be significantly altered based on its adjacent characters, i.e., by the local context - and convolutions are useful exactly in such situations where local context is paramount to the whole.}


    \part[8] Here we present a series of errors we found in the outputs of our NMT model (which is the same as the one you just trained). For each example of a reference (i.e., `gold') English translation, and NMT (i.e., `model') English translation, please:
    
    \begin{enumerate}
        \item Identify the error in the NMT translation.
        \item Provide possible reason(s) why the model may have made the error (either due to a specific linguistic construct or a specific model limitation).
        \item Describe one possible way we might alter the NMT system to fix the observed error. There are more than one possible fixes for an error. For example, it could be tweaking the size of the hidden layers or changing the attention mechanism.
    \end{enumerate}
    
    Below are the translations that you should analyze as described above. Only analyze the underlined error in each sentence. Rest assured that you don't need to know Mandarin to answer these questions. You just need to know English! If, however, you would like some additional color on the source sentences, feel free to use a resource like \url{https://www.archchinese.com/chinese_english_dictionary.html} to look up words. Feel free to search the training data file to have a better sense of how often certain characters occur.

    \begin{subparts}
        \subpart[2]
        \textbf{Source Sentence:} 贼人其后被警方拘捕及被判处盗窃罪名成立。 \newline
        \textbf{Reference Translation:} \textit{\underline{the culprits were} subsequently arrested and convicted.}\newline
        \textbf{NMT Translation:} \textit{\underline{the culprit was} subsequently arrested and sentenced to theft.}
        
        \ifans{\begin{enumerate}
            \item The NMT incorrectly translated "culprits" into the singular "culprit".
            \item Cheating with a bit of my knowledge of Chinese, plurals are not necessarily explicitly marked as such in Chinese, as they are in English (with an explicit plural form). Rather, plurality can be inferred by the rest of the sentence, which is where our model failed, most likely because it has not seen enough such contextual cues that imply plurality.
            \item The obvious solution would be to adjust our training data and add in more examples of such sentences, where plurality is only implied in the source sentence, and correctly identified in the gold translation. 
        \end{enumerate}}
            


        \subpart[2]
        \textbf{Source Sentence}: 几乎已经没有地方容纳这些人,资源已经用尽。\newline
        \textbf{Reference Translation}: \textit{there is almost no space to accommodate these people, and resources have run out.   }\newline
        \textbf{NMT Translation}: \textit{the resources have been exhausted and \underline{resources have been exhausted}.}
        
        \ifans{\begin{enumerate}
            \item The NMT repeats itself.
            \item It appears our model became too confident in this particular sequence of words, which lead to its repetition. One possible approach would be to add a heuristic that limits n-gram repetition, or, if repetition is a common occurrence add an unlikelihood objective, which can penalize the generation of already-seen tokens.
        \end{enumerate}}

        \subpart[2]
        \textbf{Source Sentence}: 当局已经宣布今天是国殇日。 \newline
        \textbf{Reference Translation}: \textit{authorities have announced \underline{a national mourning today.}}\newline
        \textbf{NMT Translation}: \textit{the administration has announced \underline{today's day.}}
        
        \ifans{\begin{enumerate}
            \item The NMT incorrectly translates "a national mourning today" to "today's day".
            \item On a hunch, I checked src.vocab and found no mentions of the "national mourning day" hanzi "国殇日" - which means the model, lacking context, will be trying to translate it in a very literal manner.
            \item Yet again, expanding of the training corpus would be a good step, or perhaps fine-tuning on a culturally-specific corpus, including commong traditions and rites.
        \end{enumerate}}
        
        \subpart[2] 
        \textbf{Source Sentence\footnote{This is a Cantonese sentence! The data used in this assignment comes from GALE Phase 3, which is a compilation of news written in simplified Chinese from various sources scraped from the internet along with their translations. For more details, see \url{https://catalog.ldc.upenn.edu/LDC2017T02}. }:} 俗语有云:``唔做唔错"。\newline
        \textbf{Reference Translation:} \textit{\underline{`` act not, err not "}, so a saying goes.}\newline
        \textbf{NMT Translation:} \textit{as the saying goes, \underline{`` it's not wrong. "}}
        
        \ifans{\begin{enumerate}
            \item The NMT mistranslates the proverb.
            \item Proverbs are notoriously difficult to translate, it is next to impossible to literally translate them into their English counterparts in a  way that makes sense if the model never explicitly saw them. And, looking through src.vocab our model indeed never this one in training.
            \item Fine-tuning on proverb translations might be in order.
        \end{enumerate}}
    \end{subparts}


    \part[14] BLEU score is the most commonly used automatic evaluation metric for NMT systems. It is usually calculated across the entire test set, but here we will consider BLEU defined for a single example.\footnote{This definition of sentence-level BLEU score matches the \texttt{sentence\_bleu()} function in the \texttt{nltk} Python package. Note that the NLTK function is sensitive to capitalization. In this question, all text is lowercased, so capitalization is irrelevant. \\ \url{http://www.nltk.org/api/nltk.translate.html\#nltk.translate.bleu_score.sentence_bleu}
    } 
    Suppose we have a source sentence $\bs$, a set of $k$ reference translations $\br_1,\dots,\br_k$, and a candidate translation $\bc$. To compute the BLEU score of $\bc$, we first compute the \textit{modified $n$-gram precision} $p_n$ of $\bc$, for each of $n=1,2,3,4$, where $n$ is the $n$ in \href{https://en.wikipedia.org/wiki/N-gram}{n-gram}:
    \begin{align}
        p_n = \frac{ \displaystyle \sum_{\text{ngram} \in \bc} \min \bigg( \max_{i=1,\dots,k} \text{Count}_{\br_i}(\text{ngram}), \enspace \text{Count}_{\bc}(\text{ngram}) \bigg) }{\displaystyle \sum_{\text{ngram}\in \bc} \text{Count}_{\bc}(\text{ngram})}
    \end{align}
     Here, for each of the $n$-grams that appear in the candidate translation $\bc$, we count the maximum number of times it appears in any one reference translation, capped by the number of times it appears in $\bc$ (this is the numerator). We divide this by the number of $n$-grams in $\bc$ (denominator). \newline 

    Next, we compute the \textit{brevity penalty} BP. Let $len(c)$ be the length of $\bc$ and let $len(r)$ be the length of the reference translation that is closest to $len(c)$ (in the case of two equally-close reference translation lengths, choose $len(r)$ as the shorter one). 
    \begin{align}
        BP = 
        \begin{cases}
            1 & \text{if } len(c) \ge len(r) \\
            \exp \big( 1 - \frac{len(r)}{len(c)} \big) & \text{otherwise}
        \end{cases}
    \end{align}
    Lastly, the BLEU score for candidate $\bc$ with respect to $\br_1,\dots,\br_k$ is:
    \begin{align}
        BLEU = BP \times \exp \Big( \sum_{n=1}^4 \lambda_n \log p_n \Big)
    \end{align}
    where $\lambda_1,\lambda_2,\lambda_3,\lambda_4$ are weights that sum to 1. The $\log$ here is natural log.
    \newline
    \begin{subparts}
        \subpart[5] Please consider this example: \newline
        Source Sentence $\bs$: \textbf{需要有充足和可预测的资源。} 
        \newline
        Reference Translation $\br_1$: \textit{resources have to be sufficient and they have to be predictable}
        \newline
        Reference Translation $\br_2$: \textit{adequate and predictable resources are required}
        
        NMT Translation $\bc_1$: there is a need for adequate and predictable resources
        
        NMT Translation $\bc_2$: resources be sufficient and predictable to
        
        Please compute the BLEU scores for $\bc_1$ and $\bc_2$. Let $\lambda_i=0.5$ for $i\in\{1,2\}$ and $\lambda_i=0$ for $i\in\{3,4\}$ (\textbf{this means we ignore 3-grams and 4-grams}, i.e., don't compute $p_3$ or $p_4$). When computing BLEU scores, show your work (i.e., show your computed values for $p_1$, $p_2$, $len(c)$, $len(r)$ and $BP$). Note that the BLEU scores can be expressed between 0 and 1 or between 0 and 100. The code is using the 0 to 100 scale while in this question we are using the \textbf{0 to 1} scale. Please round your responses to 3 decimal places. 
        \newline
        
        Which of the two NMT translations is considered the better translation according to the BLEU Score? Do you agree that it is the better translation?
        
        \ifans{For $\bc_1$:
        \begin{flalign*}
        &p_1 = \frac{4}{9}\\
        &p_2 = \frac{3}{8}\\
        &BP = \exp{(-\frac{2}{9})}\\
        &BLEU = BP \times \exp [0.5\log(\frac{4}{9}) + 0.5\log(\frac{3}{8}) = 0.327
        \end{flalign*}
        For $\bc_2$:
        \begin{flalign*}
        &p_1 = 1\\
        &p_2 = \frac{3}{5}\\
        &BP = 1\\
        &BLEU = BP \times \exp [0.5\log(1) + 0.5\log(\frac{3}{5}) = 0.775
        \end{flalign*}
        According to BLEU for 1- and 2-grams, the second translation appears a lot better, which is rather obviously not the case.}
        
        \subpart[5] Our hard drive was corrupted and we lost Reference Translation $\br_1$. Please recompute BLEU scores for $\bc_1$ and $\bc_2$, this time with respect to $\br_2$ only. Which of the two NMT translations now receives the higher BLEU score? Do you agree that it is the better translation?
        
        \ifans{For $\bc_1$:
        \begin{flalign*}
        &p_1 = \frac{4}{9}\\
        &p_2 = \frac{3}{8}\\
        &BP = 1\\
        &BLEU = BP \times \exp [0.5\log(\frac{4}{9}) + 0.5\log(\frac{3}{8}) = 0.408
        \end{flalign*}
        For $\bc_2$:
        \begin{flalign*}
        &p_1 = \frac{3}{6}\\
        &p_2 = \frac{1}{5}\\
        &BP = 1\\
        &BLEU = BP \times \exp [0.5\log(\frac{3}{6}) + 0.5\log(\frac{1}{5}) = 0.316
        \end{flalign*}
        According to BLEU for 1- and 2-grams, the first translation now appears better, which I would agree with.}
        
        \subpart[2] Due to data availability, NMT systems are often evaluated with respect to only a single reference translation. Please explain (in a few sentences) why this may be problematic. In your explanation, discuss how the BLEU score metric assesses the quality of NMT translations when there are multiple reference transitions versus a single reference translation.
        
        \ifans{As we just saw in the previous tasks, the number of reference translations can change up our BLEU results a lot - in particular $\bc_2$ "benefited" a lot from having $\br_1$ as reference. Even though in this case, removing $\br_1$ and leaving only $\br_2$ worked out for the best, it is generally not a good idea to have a single reference translation, when most sentences can easily have multiple valid translations.}
        
        \subpart[2] List two advantages and two disadvantages of BLEU, compared to human evaluation, as an evaluation metric for Machine Translation. 
        
        \ifans{Unlike human evaluation, BLEU is an objective metric, and as such, it can be easily be incorporated in model evaluation without further argumentation. It is also relatively time and resource-inexpensive compared to human evaluation. 
        
        However, the underlying domain - translation - is not necessarily objective in and of itself and it is also far more nuanced than a simple n-gram analysis can hope to capture, especially when it comes to culturally significant phrases, idioms, etc. }
        
    \end{subparts}
\end{parts}
